\documentclass[12pt]{extreport}
\usepackage{subfigure}
\usepackage[utf8]{vietnam}
\usepackage[left=3.50cm, right=2.00cm, top=3.50cm, bottom=3.00cm]{geometry}
\usepackage{fancybox,graphicx}
\usepackage{mathrsfs} 
\usepackage{amsfonts}
\usepackage{longtable,array}
\usepackage{multirow}
\newlength\mylength
\newcolumntype{C}[1]{>{\centering\arraybackslash}p{#1}}
\usepackage[intlimits]{amsmath}
\usepackage{array}
\usepackage[unicode]{hyperref}
\usepackage{algorithm}
\usepackage{algorithmicx}
\makeatletter
\renewcommand{\ALG@name}{Thuật toán}
\makeatother
\usepackage{algpseudocode}
\usepackage{amsxtra,amssymb,latexsym,amscd,amsthm}
\usepackage{enumitem}
\usepackage{tikz}
\usetikzlibrary{shapes.geometric}
\usetikzlibrary{positioning,automata}
\usepackage{scrextend}
\usepackage{longfbox}
%Môi trường Lời giải
%\newtheorem{theorem}{Định lý}[chapter]
%\newtheorem{definition}{Định nghĩa}[chapter]
%\newtheorem{example}{Ví dụ}[chapter]
%\newtheorem{lemma}[theorem]{Bổ đề}
%Tiêu đề\

\graphicspath{{./images/}}
\newtheorem{pro}{Bài toán}
\newtheorem*{constr}{Ràng buộc}
\newtheorem*{calfunc}{Các hàm được thực thi}
\newtheorem*{Sol}{Giải thuật}
\newtheorem*{Anal}{Phân tích giải thuật}

\usepackage{fancyhdr}
\pagestyle{fancy}
\lhead{}
\chead{}
\rhead{Nguyên lý Hệ Điều Hành}
\lfoot{}
\cfoot{\thepage}
\rfoot{}



\usepackage{xcolor}
\usepackage{listings}

\definecolor{mGreen}{rgb}{0,0.6,0}
\definecolor{mGray}{rgb}{0.5,0.5,0.5}
\definecolor{mPurple}{rgb}{0.58,0,0.82}
\definecolor{backgroundColour}{rgb}{0.95,0.95,0.92}

\lstdefinestyle{CStyle}{
	backgroundcolor=\color{backgroundColour},   
	commentstyle=\color{mGreen},
	keywordstyle=\color{magenta},
	numberstyle=\tiny\color{mGray},
	stringstyle=\color{mPurple},
	basicstyle=\footnotesize,
	breakatwhitespace=false,         
	breaklines=true,                 
	captionpos=b,                    
	keepspaces=true,                 
	numbers=left,                    
	numbersep=5pt,                  
	showspaces=false,                
	showstringspaces=false,
	showtabs=false,                  
	tabsize=2,
	language=C
}


\begin{document}

\thispagestyle{empty}
\thisfancypage{
	\setlength{\fboxsep}{0pt}
	\fbox}{}

\begin{center}
	
	{\fontsize{13pt}{1}\selectfont\textbf{TRƯỜNG ĐẠI HỌC BÁCH KHOA HÀ NỘI}}
	\\
	{\fontsize{13pt}{1}\selectfont\textbf{VIỆN CÔNG NGHỆ THÔNG TIN VÀ TRUYỀN THÔNG}}
	\\		
	\textbf{--------------------  o0o  ---------------------}\\[1cm]
	\includegraphics[scale=0.5]{bk_logo.png} \\[1.2cm]
\textbf{}\\[1cm]
\textbf{{\large Project 2}}\\[0.2cm]
\textbf{{\large XÂY DỰNG ỨNG DỤNG QUẢN LÝ NHÀ HÀNG CƠM BÌNH DÂN}}
\end{center}
\begin{flushleft}
\hspace{1.5 cm} \textbf{ Giáo viên hướng dẫn:\hspace{0.2cm}{ Họ và tên giảng viên}}\\[0.2cm]
\hspace{1.5 cm} \textbf{ Sinh viên thực hiện\hspace{0.3cm}:\hspace{0.2cm}{ Lê Đức Lương - 20232006P\\
\hspace{6.6 cm}Thông tin sinh viên}}\\[0.2cm]
% \hspace{1.5 cm} \textbf{ Lớp\hspace{3.6cm}:\hspace{0.2cm}{ CTTN CNTT K63}}\\
\end{flushleft}

\vfill
\begin{center}
\textbf{{\large Hà Nội - 2025}}\\
\end{center}

\tableofcontents

\newpage
\chapter*{Mở đầu}

\chapter{Khảo sát bài toán}
\section{Mục đích}
\section{Phạm vi}
\section{Yêu cầu chức năng}
\section{Yêu cầu phi chức năng}

\chapter{Cơ sở lý thuyết}

\chapter{Đặc tả yêu cầu}
\section{Sơ đồ Usecase}
\subsection{Sơ đồ Usecase tổng quát}
\subsection{Sơ đồ Usecase phân rã}
\section{Sơ đồ hoạt động}



\section{}

\subsection{}


\section{}
\subsection{}

\subsection{}

\subsection{}

\subsection{}
 
\section{}
\subsection{}
\subsection{}
\section{}
\subsection{}

\subsection{}

\subsection{}
\subsection{}

\section{}
\subsection{}

\subsection{}

\subsection{}
\subsection{}

\subsection{}

\chapter{Tổng kết}
%Báo cáo đã tổng hợp lại các bài toán khác trong đồng bộ tiến trình qua đoạn găng, bên cạnh những bài kinh điển đã được có nhiều nghiên cứu, phân tích các yêu cầu, vấn đề của từng bài và trình bày, phân tích giải thuật hiện có. Qua đó cho thấy tầm quan trọng của việc quản lý và điều độ tiến trình. Ngoài ra các bài toán được trình bày hầu hết đều sử dụng kĩ thuật đèn báo, và mỗi bài lại có những cách giải quyết khác nhau, cho thấy việc phải vận dụng kĩ thuật này linh hoạt, sáng tạo.

%Tuy nhiên các giải thuật đã trình bày có thể vẫn chưa phải là tốt nhất, và có thể có những kịch bản mà giải thuật không lường trước được, vì vậy, để bài báo cáo được hoàn thiện hơn, các thành viên trong nhóm rất mong nhận được sự đóng góp ý kiến cũng như đưa ra những nhận xét, đánh giá của thầy và các bạn. 

%tài liệu tham khảo
\newpage
\begin{thebibliography}{12}
    \addcontentsline{toc}{chapter}{\quad\  \bf Tài liệu tham khảo}
 %   \bibitem{1}The Little Books of Semaphor\\\textit{Allen B.Downey, Needham, MA, version 2.2.1}
  %  \bibitem{2}ThreadMentor: The Cigarette Smokers Problem\\ \url{https://pages.mtu.edu/~shene/NSF-3/e-Book/SEMA/TM-example-smoker.html}
   
\end{thebibliography}

\end{document}